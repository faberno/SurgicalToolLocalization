% This template was initially provided by Dulip Withanage.
% Modifications for the database systems research group
% were made by Conny Junghans,  Jannik Strötgen and Michael Gertz

\documentclass[
     12pt,         % font size
     a4paper,      % paper format
     %BCOR=10mm,     % binding correction
     DIV=14,        % stripe size for margin calculation
%     liststotoc,   % table listing in toc
%     bibtotoc,     % bibliography in toc
%     idxtotoc,     % index in toc
%     parskip       % paragraph skip instad of paragraph indent
     ]{scrreprt}

%%%%%%%%%%%%%%%%%%%%%%%%%%%%%%%%%%%%%%%%%%%%%%%%%%%%%%%%%%%%

\usepackage[parfill]{parskip}
\usepackage{amsmath,capt-of,lipsum, amssymb, bbold}
\usepackage{amsthm}
\usepackage{algorithm}
\usepackage{algpseudocode}
\usepackage{natbib}
\usepackage{graphicx}
\usepackage{caption}
\usepackage{tikz}
% \usepackage{subfig}
\usepackage{subcaption}
\usepackage[section]{placeins}
% PACKAGES:

% Use German :
\usepackage[english]{babel}
% Input and font encoding
\usepackage[utf8]{inputenc}
\usepackage[T1]{fontenc}
% Index-generation
\usepackage{makeidx}
% Einbinden von URLs:
\usepackage{url}
% Special \LaTex symbols (e.g. \BibTeX):
%\usepackage{doc}
% Include Graphic-files:
\usepackage{graphicx}
% Include doc++ generated tex-files:
%\usepackage{docxx}
% Include PDF links
%\usepackage[pdftex, bookmarks=true]{hyperref}
\usepackage{setspace}
\usepackage{sistyle}
\SIthousandsep{,}
\usepackage{array}
% hyperrefs in the documents
\usepackage[bookmarks=true,colorlinks,pdfpagelabels,pdfstartview = FitH,bookmarksopen = true,bookmarksnumbered = true,linkcolor = black,plainpages = false,hypertexnames = false,citecolor = black,urlcolor=black]{hyperref} 
%\usepackage{hyperref}

\usepackage{varwidth}
\DeclareCaptionFormat{myformat}{%
  % #1: label (e.g. "Table 1")
  % #2: separator (e.g. ": ")
  % #3: caption text
  \begin{varwidth}{\linewidth}%
    \centering
    #1#2#3%
  \end{varwidth}%
}

\usepackage{url}

%%%%%%%%%%%%%%%%%%%%%%%%%%%%%%%%%%%%%%%%%%%%%%%%%%%%%%%%%%%%

% OTHER SETTINGS:

% Pagestyle:
\pagestyle{headings}

% Choose language
\newcommand{\setlang}[1]{\selectlanguage{#1}\nonfrenchspacing}
\newcolumntype{P}[1]{>{\centering\arraybackslash}p{#1}}

\newtheorem{theorem}{Theorem}[section]
\newtheorem{definition}{Definition}[section]


\usepackage{mathtools}
\DeclarePairedDelimiter\ceil{\lceil}{\rceil}
\DeclarePairedDelimiter\floor{\lfloor}{\rfloor}

\usepackage{hyperref}

\date{\today}

\begin{document}

% TITLE:
\pagenumbering{roman} 
\begin{titlepage}


\vspace*{1cm}
\begin{center}
\vspace*{3cm}
\textbf{ 
\Large University of Heidelberg\\
\smallskip
\Large Institute of Computer Science\\
\smallskip
}

\vspace{2cm}

\textbf{\Large Project Report\\[0.4 cm]	 \large Advanced Machine Learning} % Bachelor-Arbeit 

\vspace{1.5\baselineskip}
\rule{\linewidth}{0.2 mm} \\[0.4 cm]
{\huge
\textbf{Weakly-Supervised Surgical Tool}\\
\bigskip
\textbf{Detection and Localization}
}\\
\bigskip
\rule{\linewidth}{0.2 mm} \\[1.0 cm]

by\\[0.5cm]


\begin{table}[h]
	\large
	\centering
	\begin{tabular}{P{5.5cm}P{5cm}}
		Fabian Schneider & Konrad Goldenbaum \\ 
		4017026 & 4076027 \\
	\end{tabular}
\end{table}

\vspace{0.5cm}
\includegraphics[width=0.4cm]{github}\hspace{0.15cm}\href{https://github.com/faberno/SurgicalToolLocalization}{faberno/SurgicalToolLocalization}
\vfill

\today


\end{center}

\end{titlepage}

\chapter*{Abstract}
\textit{Author: Fabian Schneider}
\begin{itemize}
	\item Bedeutung / Nutzen von Tool Lokalisierung
	\item Beschreibung des Problems
	\begin{itemize}
		\item multi-label
		\item multi-instance/same-instance
		\item weakly annotated
		\item imbalanced/fehlerhaftes dataset
	\end{itemize}
	\item Beschreibung und ziel des challenge
\end{itemize}

{\let\clearpage\relax\chapter*{Notation}}
\input{0_abstract/notation}

\newpage

% MAIN PART:
% Table of contents (dmke)
\tableofcontents
\cleardoublepage
\pagenumbering{arabic} 

% List of figures (Abbildungsverzeichnis):
%\listoffigures
% List of tables (Tabellenverzeichnis):
%\listoftables

%%%%%%%%%%%%%%%%%%%%%%%%%%%%%%%%%%%%%%%%%%%%%%%%%%%%%%%%%%%%%%%
% Here, the actual content of your thesis begins
% You can either put all the text here or use individual files to store the chapters of your thesis.
% Below are templates for both alternatives.

\chapter{Introduction}
\label{chap:intro}
what is the problem? why is it interesting? what are the main obstacles? (2 Seiten)

\section{Motivation}
\textit{Author: Konrad Goldenbaum}
\newline
Reliable surgical tool localization is a milestone on the way to applications that enable assessments of surgical performance and efficiency, identifications of the skillful use of instruments and choreographies as well as planning of operative and logistical aspects of surgical resources. In addition, the ability to automatically detect and track surgical instruments in endoscopic videos paves the way for much more complex problems such as computer-assisted or even fully automated surgery. This could foster advances in the field of telemedicine, for example, by taking on simpler parts of surgeries fully automatically, or by making the automatic delivery of instruments more fluid. These advances are only possible with the help of large data sets indicating not only the presence but also the location of instruments. This is a difficult task because it is tedious and time consuming to obtain the annotations needed to train machine learning models. This is because the bounding boxes in videos have to be labelled frame by frame so far, and for a large number of surgical tools and operations. In addition, annotators need to be trained to keep up with innovations in surgical instruments. Automated localization and classification of surgical tools makes it possible to create higher-quality datasets cost-effectively by allowing a model to learn to classify and localize instruments based on a weakly annotated dataset. Weakly annotated in this context means that for each clip only the presence of the respective tools is given.

\section{Structure}
\textit{Author: Konrad Goldenbaum}
\newline
\subsection{Objectives}

Our goal was to implement a ML-model-architecture that can be trained on different weakly-annotated endoscopic image-datasets to label and localize tools. 

\subsection{Structure}
In a first part we will describe the problem in more depth. After that we will give a short introduction into existing literature, both on image-classification as well as localization tasks, with a focus on papers that cover similar problems like the one at hand. Connected to this introduction into existing literature we cover ML-tools that should be known in order to better follow the actual contend of the work. 
In the methodology-chapter we firstly dive into the early considerations and present our first approach, since this allows us to better illustrate the challenges and discoveries made in the process. Additionally we describe the data-management-process.
We then explain the model we finally developed and the training-strategy. We finish naming our quality metrics and success-definition.
The perhaps most illustrating part, the experiments-section, helps explain the design-decisions we made on the way. Additionally, we describe in greater depth the three datasets we used. We close encouraging further work in the field.

\newpage


\chapter{Background and Related Work}
\label{chap:background}
\textit{Author: Konrad Goldenbaum}
2 Seiten

\section{Problem Description}
detailed description of the problem
\section{Related Work}
discussion of prior work

\section{ML-Tools}
introduction of relevant machine learning tools

\subsection{Convolutional Networks}
\subsection{(Variational) Auto Encoder}
\subsection{Backbones}
\subsection{Data Augmentation and Semi-Supervised Learning}
\subsection{Gaussian Process and Bayesian Optimization}
\subsection{Sliding Windows Approach}


\chapter{Methodology}
\label{chap:methodology}
\textit{Author: Konrad Goldenbaum}
detailed description of your own approach, discussion of alternatives, introduction of the experimental methodology and quality metrics


We started with the initial assumption that the project can be divided into two challenges: Firstly, the classification, and secondly the localization. We found an approach that allows for both challenges to be addressed together relatively early on. Nonetheless we will firstly cover an early alternative approach, because it will help us showcase some inherent difficulties of the challenges that had to be overcome.

We firstly decided to approach this as a problem where the images are, in a certain way, of a too high resolution. We mean this in the sense that there are tools and tissue, with the tissue obscuring the tools. We therefore followed the approach of Sun et.al (2021) to use a variational auto-encoder to reduce the images to little more than the tools, since those should be considered the most important parts of the image, or are at least the parts easiest to recognize for the De- and Encoder, since the tools appear more often than the respective tissue-structures.

Image.

Han et.al. (2017) presented a relatively straightforward VAE-model, that we decided to rebuild and adapt to our needs. 

Image (Model and Result by Huan)

Since we don't need any representation that keeps too much of the original image, we mainly trimmed the whole approach down.
In the end, so we hoped. the instruments would be the parts that would at least to some extend be "saved over" to the other side of the image. We also experimented with the code (z), hoping it might encode the information preserving locality, so that the interpretation of the code gives clues already. For this purpose we settled at around 500 latent variables, since less would not be useful in terms of placing an object within an image. 

The more straightforward approach here is of course threading the image through a needle, meaning a rather small latent variable space again, and then analysing the reconstruction. The hope was, that the image would preserve well distinguishable blobs where the tools were, so that we could filter the localization of the tools with some old-fashioned blob-detection via Otsus method or similar approaches (https://en.wikipedia.org/wiki/Otsu%27s_method).

However, although the concept of VAEs is very intriguing, there is still an issue to address: Given a successful localization of objects, how does one classify the image?
The challenge is, that even when given only the part of the image with the tool itself, this would still leave us with having to solve a rather unpleasant classification problem, since all the images come annotated with all the classes originally in the image. This problem accompanied us during the whole project. The advantage in this case is, that we would stick with the well-known one-label-classification problem, meaning that ideally only one label is true at any point, although the calculation of the loss would be difficult. We are confident that this problem would have been solvable as well. But searching for a solution for the image-classification-part, we found that there was already a better approach.

We build a first prototype of the VAE, but training was impossible at first due to the handling of the huge dataset that we used first, that comprised of over 100GB of image data. More on that in the experimentation-section. However, the sheer amount of data was not the main issue, but it points to another issue: We could not simply take the data, split it up into a training-, validation- and test-set and perform SGD on the training-set. 
This has two reasons: Firstly, the data was very unbalanced (see table 1). Secondly, and more importantly, we had to somehow make sure that the model would not learn something else, an issue that can be illustrated by some funny and some rather sinister examples (https://www.forbes.com/sites/korihale/2021/09/02/ai-bias-caused-80-of-black-mortgage-applicants-to-be-denied/).

Table: Balance of data.

To this end swapped the colour-channels randomly, flipped the image with a 50\% chance, rotated it between 0 and 90 degrees, masked and cropped the images. Masking means that we set some regions of the image to the mean value of the rest of the image. The regions where selected in the following way: Any image was divided into 24x24-tiles, with every tile having a 50\% chance of being masked. Since this is basically the binomial distribution we can expect that around half of the tiles are masked everytime. This can look like this:

masked image:

Another measure covered in the lecture on data-augmentation was the fact, that some falsely-annotated data put into the training-set might also serve the generalization-abilities of the network. We decided against any such measure, since the multi-classification of the entire clip presented us with more falsely annotated images than we might have wished for.

Having the data handled we went on exploring possible solutions to classifying the images. We therefore turned to CNNs, that, as we covered in the lecture, proved quite useful in image-classification-challenges. One interesting aspect of CNNs like the ResNET-models is that, before reducing the output of the last hidden layer to logits of the class-probabilities, these models preserve some of the size and locality of the input since the transformations are mainly convolutional in character. This means that the ResNets forward-pass can be intercepted, while the locality of the discovered features is preserved.
This also means that by computing a fully connected convolution one gets a heat-map for all the tools. We learnt about this procedure from the paper by ..., but it is fairly common already. It is quite obvious that a well-performing network that returns heat-maps of the tools presence renders any other effort to localize tools obsolete. Since this is the concept we eventually pursued, we will cover it in more detail:

The final model:
For the underlying architecture of the final model we followed a straightforward and established approach: We worked with a well-performing backbone to extract features of the image and trained a fully connected convolutional layer based off of those features. This results in a heat-map, that can be further processed.
		The intuition: Perhaps an image that shows how that is meant.
The fully connected convolutional layer replaces the sliding-window-approach. 



\chapter{Experiments}
\label{chap:experiments}
\textit{Author: Fabian Schneider}
\section{Datasets}

In the course of this project, we used 3 different datasets:

\subsection{SurgToolLoc}
The dataset of the \emph{SurgToolLoc} Challenge is provided by the \emph{International Society for Computer Aided Surgery} (ISCAS). Its a result of surgical training exercises using the \emph{da Vinci} robotic system and contains \num{24695} video clips, each 30 seconds long, captured at a resolution of 1280$\times$720 pixels and 60fps, resulting in 1800 frames per video and almost 45 Million frames in total. Each video contains 3 out of 14 different tools (\ref{fig:surgtoolloc_tools}), of which not always every tool is active and visible, resulting in a certain amount of label errors. While this training dataset is only annotated with video-level tool labels, there exists a non-public test dataset additionally annotated with bounding boxes around the robotic tools. Sadly, we could not get access to this test dataset, which is why we will only test the tool classification on it and not the localization.

\begin{figure}[h]
	\centering
	\includegraphics[width=15cm]{4_experiments/images/surgtoolloc_tools.png}
	\caption{All 14 tool classes of the \emph{SurgToolLoc} dataset.}
	\label{fig:surgtoolloc_tools}
\end{figure}

As training on such a massive dataset would have been unfeasible for our project, we created subset containing \num{10000} frames, randomly sampled from all videos and downscaled to 900$\times$620 pixels. \num{7000} of those frames were used for training and \num{3000} for testing. This re-sampling had the additional advantage of improving the extreme class-imbalance in the original data, where the most frequent class occurred \num{1000} times more often the least one (\ref{fig:surgtoolloc_tool_frequencies}).

\begin{figure}[h]
	\centering
	\includegraphics[width=15cm]{4_experiments/images/surgtoolloc_frequencies.pdf}
	\caption{Tool frequencies of the \emph{SurgToolLoc} dataset (video-level) and its subset (frame-level).}
	\label{fig:surgtoolloc_tool_frequencies}
\end{figure}

\subsection{Cholec80}
The \emph{Cholec80} dataset \cite{endonet} is provided by the \emph{CAMMA} research group at the University of Strasbourg and contains 80 videos of cholecystectomy surgeries, which combined result in about \num{180000} frames, most of them captured at resolution of 854$\times$480 pixels. It includes 7 different tool classes (\ref{fig:cholec80_tools}), from which a frame contains 0 to 3 and on average $1.32$ instances. In contrast to the \emph{SurgToolLoc} dataset, \emph{Cholec80} is annotated with frame-level tool labels, resulting in much less label errors. Bounding box labels were generated for the publication \cite{Vardazaryan}, but to our knowledge, have not been published.

\begin{figure}[h]
	\centering
	\includegraphics[width=13cm]{4_experiments/images/cholec80_tools.jpg}
	\caption{All 7 tool classes of the \emph{Cholec80} dataset.}
	\label{fig:cholec80_tools}
\end{figure}

Similar to before, this number of frames would have been unmanageable for us, so we again created a subset, of this time 6000 frames, randomly sampled from all classes. While in the original set, the factor between the least and most appearing tool is about 30, it's only 4.5 in the subset. A complete class-balance would be hard to achieve, as some of the tools, like the Grasper, almost exclusively are used with the other tools together.


\subsection{M2CAI16}

The \emph{M2CAI16} dataset contains \num{2811} frames (\num{2248} for training and \num{563} for testing) gathered from the videos 61-76 in \emph{Cholec80}. Every frame is fully annotated with tool presence labels and bounding boxes. Its size makes it manageable for us and its bounding box annotations can be used to evaluate the tool localization, which is why we will mainly use it for the following experiments.


\begin{figure}[h]
	\makebox[\linewidth][c]{
		\begin{subfigure}[b]{.6\textwidth}
			\centering
			\includegraphics[width=9cm]{4_experiments/images/cholec_frequencies.pdf}
			\label{fig:cholec80_frequncies}
		\end{subfigure}%
		\begin{subfigure}[b]{.6\textwidth}
			\centering
			\includegraphics[width=9cm]{4_experiments/images/m2cai16_frequencies.pdf}
			\label{fig:m2cai16_frequencies}
		\end{subfigure}
	}
	\caption{(left) Frame-level tool frequencies of the \emph{Cholec80} dataset and its subset \\(right) Frame-level tool frequencies of the \emph{M2CAI16} dataset}
\end{figure}

\chapter{Summary and Future Work}
\label{chap:summary}
1 Seite
Automatic localization and classification of surgical tools in endoscopic videos has many advantages, especially as an auxiliary task to provide cheaper annotated datasets for segmentation-tasks, that can ultimately improve current endeavours in medical practices and research. We managed to score a mean of over 0.95 and 0.70 average precision across all classes in detection and localization respectively on the M2CAI2016-dataset. 
These results are very satisfactory, although we hoped for better localization-results. We achieved them by combining a ResNet-18-backbone with stride 2$\times$2 with a 1$\times$1-kernel convolution-layer to generate the heatmaps for the classes and min-max-pooling for classification. 
We trained with stochastic gradient descent and batch-sizes ranging from 20-50 and the multi-label soft margin loss as the loss function.
The hyper-parameters were successfully improved over the base-parameters by applying a Gaussian process on a small set of hyper-parameters and model-performance-indicators over 50 epochs. One possible improvement here might be to combine Bayesian optimization with Gaussian processes to further improve these parameters.
We tried three pretrained networks (ResNet, AlexNet and VGGNet) as backbones, but due to extensive training we were able to show that ResNet was, for our setup, the superior choice. We were further able to decide for a 2x2-stride in the last two ResNet-convolutions as well as for min-max-pooling for classification. Taken together our architecture and training-strategy proved successful and consistent on both the M2CAI2016-dataset as well as the more complex SurgToolLoc-dataset (although we could only show that for classification in this instance).

For the future it might be up to us (or others) to try the more challenging task of phase-detection. Other possible advances might lay in segmenting the images, rather than localizing objects. Additionally, further improving localization both by using peak-responses as well as by employing other exploits is a task still ahead. Further options lay in self-supervised techniques based on generative models that might achieve considerable results without relying on annotations altogether.

\textbf{Comment:} We added a ReadMe to the repository to improve legibility of the projects code.

\chapter{Appendix}
\label{chap:appendix}
Beschreibung vom Code (2 Seiten)










\bibliographystyle{apalike}
% b) The File:
\bibliography{references}

\end{document}
